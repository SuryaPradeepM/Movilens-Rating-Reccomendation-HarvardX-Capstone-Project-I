% Options for packages loaded elsewhere
\PassOptionsToPackage{unicode}{hyperref}
\PassOptionsToPackage{hyphens}{url}
%
\documentclass[
]{article}
\usepackage{lmodern}
\usepackage{amssymb,amsmath}
\usepackage{ifxetex,ifluatex}
\ifnum 0\ifxetex 1\fi\ifluatex 1\fi=0 % if pdftex
  \usepackage[T1]{fontenc}
  \usepackage[utf8]{inputenc}
  \usepackage{textcomp} % provide euro and other symbols
\else % if luatex or xetex
  \usepackage{unicode-math}
  \defaultfontfeatures{Scale=MatchLowercase}
  \defaultfontfeatures[\rmfamily]{Ligatures=TeX,Scale=1}
\fi
% Use upquote if available, for straight quotes in verbatim environments
\IfFileExists{upquote.sty}{\usepackage{upquote}}{}
\IfFileExists{microtype.sty}{% use microtype if available
  \usepackage[]{microtype}
  \UseMicrotypeSet[protrusion]{basicmath} % disable protrusion for tt fonts
}{}
\makeatletter
\@ifundefined{KOMAClassName}{% if non-KOMA class
  \IfFileExists{parskip.sty}{%
    \usepackage{parskip}
  }{% else
    \setlength{\parindent}{0pt}
    \setlength{\parskip}{6pt plus 2pt minus 1pt}}
}{% if KOMA class
  \KOMAoptions{parskip=half}}
\makeatother
\usepackage{xcolor}
\IfFileExists{xurl.sty}{\usepackage{xurl}}{} % add URL line breaks if available
\IfFileExists{bookmark.sty}{\usepackage{bookmark}}{\usepackage{hyperref}}
\hypersetup{
  pdftitle={Movielens Rating Prediction Project},
  pdfauthor={Pradeep Kumar},
  hidelinks,
  pdfcreator={LaTeX via pandoc}}
\urlstyle{same} % disable monospaced font for URLs
\usepackage[margin=1in]{geometry}
\usepackage{color}
\usepackage{fancyvrb}
\newcommand{\VerbBar}{|}
\newcommand{\VERB}{\Verb[commandchars=\\\{\}]}
\DefineVerbatimEnvironment{Highlighting}{Verbatim}{commandchars=\\\{\}}
% Add ',fontsize=\small' for more characters per line
\usepackage{framed}
\definecolor{shadecolor}{RGB}{42,33,28}
\newenvironment{Shaded}{\begin{snugshade}}{\end{snugshade}}
\newcommand{\AlertTok}[1]{\textcolor[rgb]{1.00,1.00,0.00}{#1}}
\newcommand{\AnnotationTok}[1]{\textcolor[rgb]{0.00,0.40,1.00}{\textbf{\textit{#1}}}}
\newcommand{\AttributeTok}[1]{\textcolor[rgb]{0.74,0.68,0.62}{#1}}
\newcommand{\BaseNTok}[1]{\textcolor[rgb]{0.27,0.67,0.26}{#1}}
\newcommand{\BuiltInTok}[1]{\textcolor[rgb]{0.74,0.68,0.62}{#1}}
\newcommand{\CharTok}[1]{\textcolor[rgb]{0.02,0.61,0.04}{#1}}
\newcommand{\CommentTok}[1]{\textcolor[rgb]{0.00,0.40,1.00}{\textbf{\textit{#1}}}}
\newcommand{\CommentVarTok}[1]{\textcolor[rgb]{0.74,0.68,0.62}{#1}}
\newcommand{\ConstantTok}[1]{\textcolor[rgb]{0.74,0.68,0.62}{#1}}
\newcommand{\ControlFlowTok}[1]{\textcolor[rgb]{0.26,0.66,0.93}{\textbf{#1}}}
\newcommand{\DataTypeTok}[1]{\textcolor[rgb]{0.74,0.68,0.62}{\underline{#1}}}
\newcommand{\DecValTok}[1]{\textcolor[rgb]{0.27,0.67,0.26}{#1}}
\newcommand{\DocumentationTok}[1]{\textcolor[rgb]{0.00,0.40,1.00}{\textit{#1}}}
\newcommand{\ErrorTok}[1]{\textcolor[rgb]{1.00,1.00,0.00}{\textbf{#1}}}
\newcommand{\ExtensionTok}[1]{\textcolor[rgb]{0.74,0.68,0.62}{#1}}
\newcommand{\FloatTok}[1]{\textcolor[rgb]{0.27,0.67,0.26}{#1}}
\newcommand{\FunctionTok}[1]{\textcolor[rgb]{1.00,0.58,0.35}{\textbf{#1}}}
\newcommand{\ImportTok}[1]{\textcolor[rgb]{0.74,0.68,0.62}{#1}}
\newcommand{\InformationTok}[1]{\textcolor[rgb]{0.00,0.40,1.00}{\textbf{\textit{#1}}}}
\newcommand{\KeywordTok}[1]{\textcolor[rgb]{0.26,0.66,0.93}{\textbf{#1}}}
\newcommand{\NormalTok}[1]{\textcolor[rgb]{0.74,0.68,0.62}{#1}}
\newcommand{\OperatorTok}[1]{\textcolor[rgb]{0.74,0.68,0.62}{#1}}
\newcommand{\OtherTok}[1]{\textcolor[rgb]{0.74,0.68,0.62}{#1}}
\newcommand{\PreprocessorTok}[1]{\textcolor[rgb]{0.74,0.68,0.62}{\textbf{#1}}}
\newcommand{\RegionMarkerTok}[1]{\textcolor[rgb]{0.74,0.68,0.62}{#1}}
\newcommand{\SpecialCharTok}[1]{\textcolor[rgb]{0.02,0.61,0.04}{#1}}
\newcommand{\SpecialStringTok}[1]{\textcolor[rgb]{0.02,0.61,0.04}{#1}}
\newcommand{\StringTok}[1]{\textcolor[rgb]{0.02,0.61,0.04}{#1}}
\newcommand{\VariableTok}[1]{\textcolor[rgb]{0.74,0.68,0.62}{#1}}
\newcommand{\VerbatimStringTok}[1]{\textcolor[rgb]{0.02,0.61,0.04}{#1}}
\newcommand{\WarningTok}[1]{\textcolor[rgb]{1.00,1.00,0.00}{\textbf{#1}}}
\usepackage{longtable,booktabs}
% Correct order of tables after \paragraph or \subparagraph
\usepackage{etoolbox}
\makeatletter
\patchcmd\longtable{\par}{\if@noskipsec\mbox{}\fi\par}{}{}
\makeatother
% Allow footnotes in longtable head/foot
\IfFileExists{footnotehyper.sty}{\usepackage{footnotehyper}}{\usepackage{footnote}}
\makesavenoteenv{longtable}
\usepackage{graphicx,grffile}
\makeatletter
\def\maxwidth{\ifdim\Gin@nat@width>\linewidth\linewidth\else\Gin@nat@width\fi}
\def\maxheight{\ifdim\Gin@nat@height>\textheight\textheight\else\Gin@nat@height\fi}
\makeatother
% Scale images if necessary, so that they will not overflow the page
% margins by default, and it is still possible to overwrite the defaults
% using explicit options in \includegraphics[width, height, ...]{}
\setkeys{Gin}{width=\maxwidth,height=\maxheight,keepaspectratio}
% Set default figure placement to htbp
\makeatletter
\def\fps@figure{htbp}
\makeatother
\setlength{\emergencystretch}{3em} % prevent overfull lines
\providecommand{\tightlist}{%
  \setlength{\itemsep}{0pt}\setlength{\parskip}{0pt}}
\setcounter{secnumdepth}{5}

\title{Movielens Rating Prediction Project}
\author{Pradeep Kumar}
\date{17/06/2020}

\begin{document}
\maketitle

{
\setcounter{tocdepth}{2}
\tableofcontents
}
\hypertarget{executive-summary}{%
\section{Executive Summary}\label{executive-summary}}

The main goal of this project is to predict movie ratings using the
\textbf{MovieLens(10M)} dataset, which contains the ratings of several
movies given by various users. In this project, we start with importing,
followed by cleaning and preparing the data for analysis. Later, we
explore the dataset to find any valuable patterns. Next, we test models
to try and predict the ratings given by a specific user to a specific
movie. The train set (\texttt{edx}) is divided into a training set
(\texttt{training}) and a testing set (\texttt{testing}). The
\texttt{training} set is used to build the model and the
\texttt{testing} set is used to perform intermediate evaluations of the
model. The \texttt{validation} set is not used to train the algorithm.
The \texttt{validation} set is used only for the final RMSE evaluation.
The metrics used to evaluate the models is \textbf{Root Mean Square
Error(RMSE)} as per project guidelines. \textgreater{} We have obtained
rmse of approximately 0.864 for the best model.

\pagebreak

\hypertarget{introduction}{%
\section{Introduction}\label{introduction}}

Recommendation systems are ubiquitous nowadays. From recommending
restaurants to diagnoses, used in startups to google, recommender
systems seem to be virtually everywhere. These systems often work using
user-generated data. Large companies like Amazon that sell many products
to many customers and permit these customers to rate their products can
collect massive datasets. These datasets power their algorithms which
use online learning to predict what rating a particular user will give
to a specific item. Products are then suggested to the user based on the
predicted ratings. Recommender systems are a useful alternative to
search algorithms since they help users discover items they might not
have found otherwise. Of note, recommender systems are often implemented
using search engines indexing non-traditional data.

The vast success and implementation of recommender systems comes from
the fact that these systems can be applied to movies, news, shops,
restaurants, E-Commerce sites and so on.

For this project, we will create a movie recommendation system using the
10M version of MovieLens dataset, collected by GroupLens Research.

\hypertarget{aim-of-the-project}{%
\subsection{Aim of the project}\label{aim-of-the-project}}

The goal of this project is to train a machine learning algorithm that
predicts user ratings (on a scale of 0.5 to 5 stars) using the MovieLens
dataset split into training and validation sets to train on and predict
movie ratings the validation set.

The measure used to evaluate the algorithm's performance is the Root
Mean Square Error or RMSE. RMSE is one of the most used measures of the
differences between values predicted by a model and the observed values.
RMSE is a measure of correctness; to compare forecasting errors of
different models for a particular dataset, a lower RMSE is better than a
higher one. The effect of each error on RMSE is proportional to the size
of the squared error; Hence, more substantial errors have a
disproportionately large effect on RMSE. Consequently, RMSE is sensitive
to outliers.

The function that computes the RMSE for vectors of ratings and their
corresponding predictors will be the following:
\[ RMSE = \sqrt{\frac{1}{N}\displaystyle\sum_{u,i} (\hat{y}_{u,i}-y_{u,i})^{2}} \]

\pagebreak

\hypertarget{dataset}{%
\subsection{Dataset}\label{dataset}}

The dataset used is the
\href{https://grouplens.org/datasets/movielens/10m/}{MovieLens dataset}
of 10,000,054 ratings applied to 10,681 movies by 71,567 users of the
online movie recommender service MovieLens. Users were selected at
random for inclusion. All users had rated at least 20 movies.

The Script to download and create the training(edx) and validation sets:
- \href{https://grouplens.org/datasets/movielens/10m/}{MovieLens 10M
dataset} -
\href{http://files.grouplens.org/datasets/movielens/ml-10m.zip}{MovieLens
10M dataset - zip file}

\begin{Shaded}
\begin{Highlighting}[]
\CommentTok{################################}
\CommentTok{# Create edx set, validation set}
\CommentTok{################################}

\CommentTok{# MovieLens 10M dataset:}
\CommentTok{# https://grouplens.org/datasets/movielens/10m/}
\CommentTok{# http://files.grouplens.org/datasets/movielens/ml-10m.zip}

\CommentTok{# Note: this process could take a couple of minutes}
\NormalTok{dl <-}\StringTok{ }\KeywordTok{tempfile}\NormalTok{()}
\KeywordTok{download.file}\NormalTok{(}\StringTok{"http://files.grouplens.org/datasets/movielens/ml-10m.zip"}\NormalTok{, dl)}

\NormalTok{ratings <-}\StringTok{ }\KeywordTok{fread}\NormalTok{(}\DataTypeTok{text =} \KeywordTok{gsub}\NormalTok{(}\StringTok{"::"}\NormalTok{, }\StringTok{"}\CharTok{\textbackslash{}t}\StringTok{"}\NormalTok{, }\KeywordTok{readLines}\NormalTok{(}\KeywordTok{unzip}\NormalTok{(dl, }\StringTok{"ml-10M100K/ratings.dat"}\NormalTok{))),}
                 \DataTypeTok{col.names =} \KeywordTok{c}\NormalTok{(}\StringTok{"userId"}\NormalTok{, }\StringTok{"movieId"}\NormalTok{, }\StringTok{"rating"}\NormalTok{, }\StringTok{"timestamp"}\NormalTok{))}

\NormalTok{movies <-}\StringTok{ }\KeywordTok{str_split_fixed}\NormalTok{(}\KeywordTok{readLines}\NormalTok{(}\KeywordTok{unzip}\NormalTok{(dl, }\StringTok{"ml-10M100K/movies.dat"}\NormalTok{)), }\StringTok{"}\CharTok{\textbackslash{}\textbackslash{}}\StringTok{::"}\NormalTok{, }\DecValTok{3}\NormalTok{)}
\KeywordTok{colnames}\NormalTok{(movies) <-}\StringTok{ }\KeywordTok{c}\NormalTok{(}\StringTok{"movieId"}\NormalTok{, }\StringTok{"title"}\NormalTok{, }\StringTok{"genres"}\NormalTok{)}

\NormalTok{movies <-}\StringTok{ }\KeywordTok{as.data.frame}\NormalTok{(movies) }\OperatorTok\StringTok{ }\KeywordTok{mutate}\NormalTok{(}\DataTypeTok{movieId =} \KeywordTok{as.numeric}\NormalTok{(movieId),}
                                           \DataTypeTok{title =} \KeywordTok{as.character}\NormalTok{(title),}
                                           \DataTypeTok{genres =} \KeywordTok{as.character}\NormalTok{(genres))}

\NormalTok{movielens <-}\StringTok{ }\KeywordTok{left_join}\NormalTok{(ratings, movies, }\DataTypeTok{by =} \StringTok{"movieId"}\NormalTok{)}
\end{Highlighting}
\end{Shaded}

We split the dataset into train(edx) and validation sets. Model training
is carried out on the ``edx'' subset only, ``validation'' subset will be
used only to test the trained model.

\begin{Shaded}
\begin{Highlighting}[]
\CommentTok{# Validation set will be 10% of MovieLens data}
\KeywordTok{set.seed}\NormalTok{(}\DecValTok{1}\NormalTok{, }\DataTypeTok{sample.kind=}\StringTok{"Rounding"}\NormalTok{)}
\CommentTok{# if using R 3.5 or earlier, use `set.seed(1)` instead}
\NormalTok{test_index <-}\StringTok{ }\KeywordTok{createDataPartition}\NormalTok{(}\DataTypeTok{y =}\NormalTok{ movielens}\OperatorTok{$}\NormalTok{rating, }\DataTypeTok{times =} \DecValTok{1}\NormalTok{, }\DataTypeTok{p =} \FloatTok{0.1}\NormalTok{, }\DataTypeTok{list =} \OtherTok{FALSE}\NormalTok{)}
\NormalTok{edx <-}\StringTok{ }\NormalTok{movielens[}\OperatorTok{-}\NormalTok{test_index,]}
\NormalTok{temp <-}\StringTok{ }\NormalTok{movielens[test_index,]}

\CommentTok{# Make sure userId and movieId in validation set are also in edx set}
\NormalTok{validation <-}\StringTok{ }\NormalTok{temp }\OperatorTok\StringTok{ }
\StringTok{  }\KeywordTok{semi_join}\NormalTok{(edx, }\DataTypeTok{by =} \StringTok{"movieId"}\NormalTok{) }\OperatorTok
\StringTok{  }\KeywordTok{semi_join}\NormalTok{(edx, }\DataTypeTok{by =} \StringTok{"userId"}\NormalTok{)}

\CommentTok{# Add rows removed from validation set back into edx set}
\NormalTok{removed <-}\StringTok{ }\KeywordTok{anti_join}\NormalTok{(temp, validation)}
\NormalTok{edx <-}\StringTok{ }\KeywordTok{rbind}\NormalTok{(edx, removed)}

\CommentTok{# Delete the unnecessary dataframes created to free up memory}
\KeywordTok{rm}\NormalTok{(dl, ratings, movies, test_index, temp, movielens, removed)}
\end{Highlighting}
\end{Shaded}

\pagebreak

\hypertarget{exploratory-data-analysis}{%
\section{Exploratory Data Analysis}\label{exploratory-data-analysis}}

In this section we explore the data in \texttt{edx}, try to find
patterns and insights to inform further steps of the data science cycle.

\hypertarget{basic-information}{%
\subsection{Basic Information}\label{basic-information}}

First, To get familiar with the dataset, we look at the head of the
dataset. The subset contains the six variables: \texttt{userID},
\texttt{movieID}, \texttt{rating}, \texttt{timestamp}, \texttt{title},
and \texttt{genres}. We also note that the data is in tidy format--Each
row represents a single rating of a user for a single movie.

\begin{verbatim}
  userId movieId rating timestamp                         title
1      1     122      5 838985046              Boomerang (1992)
2      1     185      5 838983525               Net, The (1995)
4      1     292      5 838983421               Outbreak (1995)
5      1     316      5 838983392               Stargate (1994)
6      1     329      5 838983392 Star Trek: Generations (1994)
7      1     355      5 838984474       Flintstones, The (1994)
                         genres
1                Comedy|Romance
2         Action|Crime|Thriller
4  Action|Drama|Sci-Fi|Thriller
5       Action|Adventure|Sci-Fi
6 Action|Adventure|Drama|Sci-Fi
7       Children|Comedy|Fantasy
\end{verbatim}

A summary of the subset confirms that there are no missing values(NAs).
Hence, we neend not bother ourselves with removing or imputing missing
values.

\begin{verbatim}
     userId         movieId          rating        timestamp        
 Min.   :    1   Min.   :    1   Min.   :0.500   Min.   :7.897e+08  
 1st Qu.:18124   1st Qu.:  648   1st Qu.:3.000   1st Qu.:9.468e+08  
 Median :35738   Median : 1834   Median :4.000   Median :1.035e+09  
 Mean   :35870   Mean   : 4122   Mean   :3.512   Mean   :1.033e+09  
 3rd Qu.:53607   3rd Qu.: 3626   3rd Qu.:4.000   3rd Qu.:1.127e+09  
 Max.   :71567   Max.   :65133   Max.   :5.000   Max.   :1.231e+09  
    title              genres         
 Length:9000055     Length:9000055    
 Class :character   Class :character  
 Mode  :character   Mode  :character  
                                      
                                      
                                      
\end{verbatim}

\begin{itemize}
\tightlist
\item
  Unique users and movies: The number of unique users in the edx
  dataframe is about 70,000 and it contains ratings for about 10,700
  unique movies.
\end{itemize}

\begin{Shaded}
\begin{Highlighting}[]
\CommentTok{# Number of unique users & movies present in the datset}
\NormalTok{edx }\OperatorTok\StringTok{ }\KeywordTok{summarize}\NormalTok{(}\DataTypeTok{n_users =} \KeywordTok{n_distinct}\NormalTok{(userId), }
                  \DataTypeTok{n_movies =} \KeywordTok{n_distinct}\NormalTok{(movieId))}
\end{Highlighting}
\end{Shaded}

\begin{verbatim}
  n_users n_movies
1   69878    10677
\end{verbatim}

\pagebreak

\begin{itemize}
\tightlist
\item
  Highest rated Movies:
\end{itemize}

\begin{Shaded}
\begin{Highlighting}[]
\CommentTok{# Highest Rated movies}
\NormalTok{edx }\OperatorTok\StringTok{ }\KeywordTok{group_by}\NormalTok{(title) }\OperatorTok
\StringTok{        }\KeywordTok{summarize}\NormalTok{(}\DataTypeTok{numberOfRatings =} \KeywordTok{n}\NormalTok{(), }\DataTypeTok{averageRating =} \KeywordTok{mean}\NormalTok{(rating)) }\OperatorTok
\StringTok{        }\KeywordTok{arrange}\NormalTok{(}\KeywordTok{desc}\NormalTok{(averageRating)) }\OperatorTok
\StringTok{        }\KeywordTok{top_n}\NormalTok{(}\DecValTok{10}\NormalTok{, }\DataTypeTok{wt=}\NormalTok{averageRating)}
\end{Highlighting}
\end{Shaded}

\begin{verbatim}
# A tibble: 10 x 3
   title                                           numberOfRatings averageRating
   <chr>                                                     <int>         <dbl>
 1 Blue Light, The (Das Blaue Licht) (1932)                      1          5   
 2 Fighting Elegy (Kenka erejii) (1966)                          1          5   
 3 Hellhounds on My Trail (1999)                                 1          5   
 4 Satan's Tango (Sátántangó) (1994)                          2          5   
 5 Shadows of Forgotten Ancestors (1964)                         1          5   
 6 Sun Alley (Sonnenallee) (1999)                                1          5   
 7 Constantine's Sword (2007)                                    2          4.75
 8 Human Condition II, The (Ningen no joken II) (~               4          4.75
 9 Human Condition III, The (Ningen no joken III)~               4          4.75
10 Who's Singin' Over There? (a.k.a. Who Sings Ov~               4          4.75
\end{verbatim}

These highest rated movies are very obscure movies. It can be noted that
the number of ratings for these movies is meagre, in some cases only a
single rating. To find a more fair list of highest rated movies, we need
to take into amount the number of ratings. The following shows a list of
the highest-rated movies that at least 100 ratings.

\begin{Shaded}
\begin{Highlighting}[]
\CommentTok{# Highest Rated movies with atleast 100 ratings}
\NormalTok{edx }\OperatorTok\StringTok{ }\KeywordTok{group_by}\NormalTok{(title) }\OperatorTok
\StringTok{        }\KeywordTok{summarize}\NormalTok{(}\DataTypeTok{numberOfRatings =} \KeywordTok{n}\NormalTok{(), }\DataTypeTok{averageRating =} \KeywordTok{mean}\NormalTok{(rating)) }\OperatorTok
\StringTok{        }\KeywordTok{filter}\NormalTok{(numberOfRatings }\OperatorTok{>}\StringTok{ }\DecValTok{100}\NormalTok{) }\OperatorTok
\StringTok{        }\KeywordTok{arrange}\NormalTok{(}\KeywordTok{desc}\NormalTok{(averageRating)) }\OperatorTok
\StringTok{        }\KeywordTok{top_n}\NormalTok{(}\DecValTok{10}\NormalTok{, }\DataTypeTok{wt=}\NormalTok{averageRating)}
\end{Highlighting}
\end{Shaded}

\begin{verbatim}
# A tibble: 10 x 3
   title                                         numberOfRatings averageRating
   <chr>                                                   <int>         <dbl>
 1 Shawshank Redemption, The (1994)                        28015          4.46
 2 Godfather, The (1972)                                   17747          4.42
 3 Usual Suspects, The (1995)                              21648          4.37
 4 Schindler's List (1993)                                 23193          4.36
 5 Casablanca (1942)                                       11232          4.32
 6 Rear Window (1954)                                       7935          4.32
 7 Sunset Blvd. (a.k.a. Sunset Boulevard) (1950)            2922          4.32
 8 Third Man, The (1949)                                    2967          4.31
 9 Double Indemnity (1944)                                  2154          4.31
10 Paths of Glory (1957)                                    1571          4.31
\end{verbatim}

Each movie is assigned one or more genres, and the genres are encoded
into one field genres. The different genres and the number of movies
they are assigned to are as folows.

\begin{Shaded}
\begin{Highlighting}[]
\CommentTok{# Extract unique genres with seperate_rows and arrange them in descending order}
\NormalTok{genres <-}\StringTok{ }\NormalTok{edx }\OperatorTok\StringTok{ }\KeywordTok{separate_rows}\NormalTok{(genres, }\DataTypeTok{sep =} \StringTok{"}\CharTok{\textbackslash{}\textbackslash{}}\StringTok{|"}\NormalTok{) }\OperatorTok
\StringTok{  }\KeywordTok{group_by}\NormalTok{(genres) }\OperatorTok
\StringTok{  }\KeywordTok{summarize}\NormalTok{(}\DataTypeTok{n =} \KeywordTok{n}\NormalTok{()) }\OperatorTok
\StringTok{  }\KeywordTok{arrange}\NormalTok{(}\KeywordTok{desc}\NormalTok{(n))}

\NormalTok{genres }\OperatorTok\StringTok{ }\KeywordTok{print.data.frame}\NormalTok{()}
\end{Highlighting}
\end{Shaded}

\begin{verbatim}
               genres       n
1               Drama 3910127
2              Comedy 3540930
3              Action 2560545
4            Thriller 2325899
5           Adventure 1908892
6             Romance 1712100
7              Sci-Fi 1341183
8               Crime 1327715
9             Fantasy  925637
10           Children  737994
11             Horror  691485
12            Mystery  568332
13                War  511147
14          Animation  467168
15            Musical  433080
16            Western  189394
17          Film-Noir  118541
18        Documentary   93066
19               IMAX    8181
20 (no genres listed)       7
\end{verbatim}

We note that the dataset contains \textbf{20} different genres and a
pseudo-genre called (no genres listed) indicating that the movie has not
been assigned any genres.

\pagebreak

\hypertarget{visuals}{%
\subsection{Visuals}\label{visuals}}

We note that there are higher compared to lower ones as seen by the
distribution of ratings shown below. This discrepancy may be because
people tend to rate movies they love or hate and do not bother to rate
the movies they feel are average. Further, We observe that 4 is the most
common rating, followed by 3 and 5. 0.5 is the least common rating. In
general, half-a-star ratings are less common than whole star ratings.

\includegraphics{Movielens_Predictions_files/figure-latex/rating_distribution-1.pdf}

We observe that some movies are rated by more users compared to others,
while some have very few ratings, and some have only one rating. One
hundred twenty-five movies have ratings from a single user. This fact is
important to note as deficient ratings might result in our model not
generalizing well, leading to inaccurate estimates for our predictions.

So, we consider introducing \textbf{Regularization} later.
Regularizations are techniques used to reduce the error by fitting a
function appropriately on the given training set and avoid overfitting
(the production of an analysis that corresponds too closely or exactly
to a particular set of data and may, therefore, fail to fit additional
data or predict future observations reliably). A penalty term is added
to the cost function. The additional term controls the excessively
fluctuating function such that the coefficients do not take extreme
values. For this reason, Adding Regularisation and a penalty term to the
models is expected to lead to better models as it penalizes features.

\begin{Shaded}
\begin{Highlighting}[]
\CommentTok{# Generate log scaled frequency distribution of movies and ratings}
\NormalTok{edx }\OperatorTok\StringTok{ }\KeywordTok{count}\NormalTok{(movieId) }\OperatorTok
\StringTok{  }\KeywordTok{ggplot}\NormalTok{(}\KeywordTok{aes}\NormalTok{(n)) }\OperatorTok{+}
\StringTok{  }\KeywordTok{geom_histogram}\NormalTok{(}\DataTypeTok{bins =} \DecValTok{50}\NormalTok{, }\DataTypeTok{color =} \StringTok{"#51A8C9"}\NormalTok{) }\OperatorTok{+}
\StringTok{  }\KeywordTok{scale_x_log10}\NormalTok{() }\OperatorTok{+}
\StringTok{  }\KeywordTok{ggtitle}\NormalTok{(}\StringTok{"Frequency distribution of Ratings for Movies"}\NormalTok{) }\OperatorTok{+}\StringTok{ }
\StringTok{  }\KeywordTok{scale_colour_wsj}\NormalTok{(}\StringTok{"colors6"}\NormalTok{, }\StringTok{""}\NormalTok{) }\OperatorTok{+}
\StringTok{  }\KeywordTok{theme_wsj}\NormalTok{(}\DataTypeTok{base_size =} \DecValTok{5}\NormalTok{, }\DataTypeTok{color =} \StringTok{"blue"}\NormalTok{, }
            \DataTypeTok{base_family =} \StringTok{"sans"}\NormalTok{, }\DataTypeTok{title_family =} \StringTok{"sans"}\NormalTok{)}
\end{Highlighting}
\end{Shaded}

\includegraphics{Movielens_Predictions_files/figure-latex/number_of_ratings_per_movie-1.pdf}

\pagebreak

These obscure movies are outliers which have been rated only once by
user, Predictions of future ratings for them might prove to be
difficult.

\begin{Shaded}
\begin{Highlighting}[]
\CommentTok{# Find the movies with only a single user rating (Outliers)}
\NormalTok{edx }\OperatorTok\StringTok{ }\KeywordTok{group_by}\NormalTok{(movieId) }\OperatorTok
\StringTok{  }\KeywordTok{summarize}\NormalTok{(}\DataTypeTok{ratings =} \KeywordTok{n}\NormalTok{()) }\OperatorTok
\StringTok{  }\KeywordTok{filter}\NormalTok{(ratings }\OperatorTok{==}\StringTok{ }\DecValTok{1}\NormalTok{) }\OperatorTok
\StringTok{  }\KeywordTok{left_join}\NormalTok{(edx, }\DataTypeTok{by =} \StringTok{"movieId"}\NormalTok{) }\OperatorTok
\StringTok{  }\KeywordTok{group_by}\NormalTok{(title) }\OperatorTok
\StringTok{  }\KeywordTok{summarize}\NormalTok{(}\DataTypeTok{rating =}\NormalTok{ rating, }\DataTypeTok{n_rating =}\NormalTok{ ratings) }\OperatorTok
\StringTok{  }\KeywordTok{slice}\NormalTok{(}\DecValTok{1}\OperatorTok{:}\DecValTok{20}\NormalTok{) }\OperatorTok
\StringTok{  }\NormalTok{knitr}\OperatorTok{::}\KeywordTok{kable}\NormalTok{()}
\end{Highlighting}
\end{Shaded}

\begin{longtable}[]{@{}lrr@{}}
\toprule
title & rating & n\_rating\tabularnewline
\midrule
\endhead
1, 2, 3, Sun (Un, deuz, trois, soleil) (1993) & 2.0 & 1\tabularnewline
100 Feet (2008) & 2.0 & 1\tabularnewline
4 (2005) & 2.5 & 1\tabularnewline
Accused (Anklaget) (2005) & 0.5 & 1\tabularnewline
Ace of Hearts (2008) & 2.0 & 1\tabularnewline
Ace of Hearts, The (1921) & 3.5 & 1\tabularnewline
Adios, Sabata (Indio Black, sai che ti dico: Sei un gran figlio
di\ldots) (1971) & 1.5 & 1\tabularnewline
Africa addio (1966) & 3.0 & 1\tabularnewline
Aleksandra (2007) & 3.0 & 1\tabularnewline
Bad Blood (Mauvais sang) (1986) & 4.5 & 1\tabularnewline
Battle of Russia, The (Why We Fight, 5) (1943) & 3.5 & 1\tabularnewline
Bellissima (1951) & 4.0 & 1\tabularnewline
Big Fella (1937) & 3.0 & 1\tabularnewline
Black Tights (1-2-3-4 ou Les Collants noirs) (1960) & 3.0 &
1\tabularnewline
Blind Shaft (Mang jing) (2003) & 2.5 & 1\tabularnewline
Blue Light, The (Das Blaue Licht) (1932) & 5.0 & 1\tabularnewline
Borderline (1950) & 3.0 & 1\tabularnewline
Brothers of the Head (2005) & 2.5 & 1\tabularnewline
Chapayev (1934) & 1.5 & 1\tabularnewline
Cold Sweat (De la part des copains) (1970) & 2.5 & 1\tabularnewline
\bottomrule
\end{longtable}

\pagebreak

We observe that the majority of users have rated only between 30 and 100
movies. So, a user penalty term needs to be included later in our models
to account for this.

\begin{Shaded}
\begin{Highlighting}[]
\CommentTok{# Generate the log scaled frequency distribution of users and ratings}
\NormalTok{edx }\OperatorTok\StringTok{ }\KeywordTok{count}\NormalTok{(userId) }\OperatorTok
\StringTok{  }\KeywordTok{ggplot}\NormalTok{(}\KeywordTok{aes}\NormalTok{(n)) }\OperatorTok{+}
\StringTok{  }\KeywordTok{geom_histogram}\NormalTok{(}\DataTypeTok{bins =} \DecValTok{50}\NormalTok{, }\DataTypeTok{color =} \StringTok{"#51A8C9"}\NormalTok{) }\OperatorTok{+}
\StringTok{  }\KeywordTok{scale_x_log10}\NormalTok{() }\OperatorTok{+}
\StringTok{  }\KeywordTok{ggtitle}\NormalTok{(}\StringTok{"Frequency distribution of Ratings for Users"}\NormalTok{) }\OperatorTok{+}
\StringTok{  }\KeywordTok{scale_colour_wsj}\NormalTok{(}\StringTok{"colors6"}\NormalTok{, }\StringTok{""}\NormalTok{) }\OperatorTok{+}
\StringTok{  }\KeywordTok{theme_wsj}\NormalTok{(}\DataTypeTok{base_size =} \DecValTok{5}\NormalTok{, }\DataTypeTok{color =} \StringTok{"blue"}\NormalTok{, }
            \DataTypeTok{base_family =} \StringTok{"sans"}\NormalTok{, }\DataTypeTok{title_family =} \StringTok{"sans"}\NormalTok{)}
\end{Highlighting}
\end{Shaded}

\includegraphics{Movielens_Predictions_files/figure-latex/number_ratings_given_by_users-1.pdf}

Also, Users differ vastly in how critical they are with their ratings.
Some users tend to give much lower ratings, and some users tend to give
higher ratings than average. The graph below only includes users that
have rated at least 100 movies.

\begin{Shaded}
\begin{Highlighting}[]
\CommentTok{# Generate mean ratings of users who have rated atleast 100 movies}
\NormalTok{edx }\OperatorTok\StringTok{ }\KeywordTok{group_by}\NormalTok{(userId) }\OperatorTok
\StringTok{  }\KeywordTok{filter}\NormalTok{(}\KeywordTok{n}\NormalTok{() }\OperatorTok{>=}\StringTok{ }\DecValTok{100}\NormalTok{) }\OperatorTok
\StringTok{  }\KeywordTok{summarize}\NormalTok{(}\DataTypeTok{mean_rating =} \KeywordTok{mean}\NormalTok{(rating)) }\OperatorTok
\StringTok{  }\KeywordTok{ggplot}\NormalTok{(}\KeywordTok{aes}\NormalTok{(mean_rating)) }\OperatorTok{+}
\StringTok{  }\KeywordTok{geom_histogram}\NormalTok{(}\DataTypeTok{bins =} \DecValTok{40}\NormalTok{, }\DataTypeTok{color =} \StringTok{"#51A8C9"}\NormalTok{) }\OperatorTok{+}
\StringTok{  }\KeywordTok{xlab}\NormalTok{(}\StringTok{"Average rating"}\NormalTok{) }\OperatorTok{+}
\StringTok{  }\KeywordTok{ylab}\NormalTok{(}\StringTok{"Count of users"}\NormalTok{) }\OperatorTok{+}
\StringTok{  }\KeywordTok{ggtitle}\NormalTok{(}\StringTok{"Average Ratings by Number of Users"}\NormalTok{) }\OperatorTok{+}
\StringTok{  }\KeywordTok{scale_x_discrete}\NormalTok{(}\DataTypeTok{limits =} \KeywordTok{c}\NormalTok{(}\KeywordTok{seq}\NormalTok{(}\FloatTok{0.5}\NormalTok{,}\DecValTok{5}\NormalTok{,}\FloatTok{0.5}\NormalTok{))) }\OperatorTok{+}
\StringTok{  }\KeywordTok{scale_colour_wsj}\NormalTok{(}\StringTok{"colors6"}\NormalTok{, }\StringTok{""}\NormalTok{) }\OperatorTok{+}
\StringTok{  }\KeywordTok{theme_wsj}\NormalTok{(}\DataTypeTok{base_size =} \DecValTok{5}\NormalTok{, }\DataTypeTok{color =} \StringTok{"blue"}\NormalTok{, }
            \DataTypeTok{base_family =} \StringTok{"sans"}\NormalTok{, }\DataTypeTok{title_family =} \StringTok{"sans"}\NormalTok{)}
\end{Highlighting}
\end{Shaded}

\includegraphics{Movielens_Predictions_files/figure-latex/Mean_movie_ratings_given_by_users-1.pdf}

\pagebreak

\hypertarget{models-and-analysis}{%
\section{Models and Analysis}\label{models-and-analysis}}

We write now the loss-function, previously explained, that computes the
RMSE:

\[ RMSE = \sqrt{\frac{1}{N}\displaystyle\sum_{u,i} (\hat{y}_{u,i}-y_{u,i})^{2}} \]

The RMSE is our evaluation metric for the models. The lower the RMSE,
the better our model is. The function we use to compute the RMSE for
vectors of ratings, and their corresponding predictions is as follows.
Where N is the number of user/movie combinations, and the sum of squared
errors is calculated over all these combinations.

\begin{Shaded}
\begin{Highlighting}[]
\CommentTok{# Root of mean of squared errors}
\NormalTok{RMSE <-}\StringTok{ }\ControlFlowTok{function}\NormalTok{(true_ratings, predicted_ratings)\{}
  \KeywordTok{sqrt}\NormalTok{(}\KeywordTok{mean}\NormalTok{((true_ratings }\OperatorTok{-}\StringTok{ }\NormalTok{predicted_ratings)}\OperatorTok{^}\DecValTok{2}\NormalTok{))}
\NormalTok{\}}
\end{Highlighting}
\end{Shaded}

\hypertarget{nauxefve-average-movie-rating-model}{%
\subsection{Naïve Average movie rating
model}\label{nauxefve-average-movie-rating-model}}

The first basic model predicts the same rating for all movies, so we
compute the dataset's mean rating. The expected rating of the underlying
data set is between 3 and 4. This model simply always predicts the
average of all the ratings.

\[ Y_{u, i} = \mu + \epsilon_{u, i} \] where \(\epsilon_{u,i}\)
independent error sample from the same distribution centered at 0 and
\(\mu\) the average rating for all movies. This very simple model makes
the assumption that all differences in movie ratings are explained by
random variation alone. We know that the estimate that minimize the RMSE
is the least square estimate of \(Y_{u,i}\) , in this case, is the
average of all ratings:

\begin{Shaded}
\begin{Highlighting}[]
\NormalTok{mu <-}\StringTok{ }\KeywordTok{mean}\NormalTok{(edx}\OperatorTok{$}\NormalTok{rating)}
\NormalTok{mu}
\end{Highlighting}
\end{Shaded}

\begin{verbatim}
[1] 3.512465
\end{verbatim}

If we predict all unknown ratings with \(\mu\) or mu, we obtain the
first naive RMSE:

\begin{Shaded}
\begin{Highlighting}[]
\CommentTok{# Rmse for average rating model}
\NormalTok{naive_rmse <-}\StringTok{ }\KeywordTok{RMSE}\NormalTok{(validation}\OperatorTok{$}\NormalTok{rating, mu)}
\NormalTok{naive_rmse}
\end{Highlighting}
\end{Shaded}

\begin{verbatim}
[1] 1.061202
\end{verbatim}

\pagebreak

Tabulated as follows:

\begin{Shaded}
\begin{Highlighting}[]
\NormalTok{rmse_results <-}\StringTok{ }\KeywordTok{tibble}\NormalTok{(}\DataTypeTok{Model =} \StringTok{"Naïve Average movie rating model"}\NormalTok{, }\DataTypeTok{RMSE =}\NormalTok{ naive_rmse)}
\NormalTok{rmse_results }\OperatorTok\StringTok{ }\NormalTok{knitr}\OperatorTok{::}\KeywordTok{kable}\NormalTok{()}
\end{Highlighting}
\end{Shaded}

\begin{longtable}[]{@{}lr@{}}
\toprule
Model & RMSE\tabularnewline
\midrule
\endhead
Naïve Average movie rating model & 1.061202\tabularnewline
\bottomrule
\end{longtable}

This give us our baseline RMSE to compare with next modelling
approaches.

In order to do better than simply predicting the average rating, we
incorporate some of insights we gained during the exploratory data
analysis.

\hypertarget{movie-effect-model}{%
\subsection{Movie effect model}\label{movie-effect-model}}

To improve above model we focus on the fact that, from experience, we
know that some movies are just generally rated higher than others.
Higher ratings are mostly linked to popular movies among users and the
opposite is true for unpopular movies. We compute the estimated
deviation of each movie's mean rating from the total mean of all movies
\(\mu\). The resulting variable is called ``b'' ( as bias ) for each
movie ``i'' \(b_{i}\), that represents average ranking for movie \(i\):
\[Y_{u, i} = \mu +b_{i}+ \epsilon_{u, i}\]

The distribution is skewed, implying that more movies have negative
effects

\begin{Shaded}
\begin{Highlighting}[]
\CommentTok{# Compute the bias terms of movie ratings}
\NormalTok{movie_avgs <-}\StringTok{ }\NormalTok{edx }\OperatorTok
\StringTok{  }\KeywordTok{group_by}\NormalTok{(movieId) }\OperatorTok
\StringTok{  }\KeywordTok{summarize}\NormalTok{(}\DataTypeTok{b_i =} \KeywordTok{mean}\NormalTok{(rating }\OperatorTok{-}\StringTok{ }\NormalTok{mu))}

\CommentTok{# plot distribution}
\NormalTok{movie_avgs }\OperatorTok\StringTok{ }\KeywordTok{qplot}\NormalTok{(b_i, }\DataTypeTok{geom =}\StringTok{"histogram"}\NormalTok{, }\DataTypeTok{bins =} \DecValTok{10}\NormalTok{, }\DataTypeTok{data =}\NormalTok{ ., }\DataTypeTok{color =} \KeywordTok{I}\NormalTok{(}\StringTok{"#51A8C9"}\NormalTok{),}
                    \DataTypeTok{main =} \StringTok{"Number of movies for respective computed bias"}\NormalTok{) }\OperatorTok{+}\StringTok{ }
\StringTok{               }\KeywordTok{scale_colour_wsj}\NormalTok{(}\StringTok{"colors6"}\NormalTok{, }\StringTok{""}\NormalTok{) }\OperatorTok{+}
\StringTok{               }\KeywordTok{theme_wsj}\NormalTok{(}\DataTypeTok{base_size =} \DecValTok{5}\NormalTok{, }\DataTypeTok{color =} \StringTok{"blue"}\NormalTok{, }
                        \DataTypeTok{base_family =} \StringTok{"sans"}\NormalTok{, }\DataTypeTok{title_family =} \StringTok{"sans"}\NormalTok{)}
\end{Highlighting}
\end{Shaded}

\includegraphics{Movielens_Predictions_files/figure-latex/Number_of_movies_with_the computed_b_i-1.pdf}

We incorporate the movie effect term to our prediction:

\begin{Shaded}
\begin{Highlighting}[]
\CommentTok{# Left join moive_avgs on key movieId}
\NormalTok{predictions <-}\StringTok{ }\NormalTok{mu }\OperatorTok{+}\StringTok{  }\NormalTok{validation }\OperatorTok
\StringTok{                     }\KeywordTok{left_join}\NormalTok{(movie_avgs, }\DataTypeTok{by=}\StringTok{'movieId'}\NormalTok{) }\OperatorTok
\StringTok{                     }\KeywordTok{pull}\NormalTok{(b_i)}

\NormalTok{movie_effect_rmse <-}\StringTok{ }\KeywordTok{RMSE}\NormalTok{(predictions, validation}\OperatorTok{$}\NormalTok{rating)}
\NormalTok{rmse_results <-}\StringTok{ }\KeywordTok{bind_rows}\NormalTok{(rmse_results,}
                          \KeywordTok{tibble}\NormalTok{(}\DataTypeTok{Model=}\StringTok{"Movie effect model"}\NormalTok{,  }
                                     \DataTypeTok{RMSE =}\NormalTok{ movie_effect_rmse ))}
\NormalTok{rmse_results }\OperatorTok\StringTok{ }\NormalTok{knitr}\OperatorTok{::}\KeywordTok{kable}\NormalTok{()}
\end{Highlighting}
\end{Shaded}

\begin{longtable}[]{@{}lr@{}}
\toprule
Model & RMSE\tabularnewline
\midrule
\endhead
Naïve Average movie rating model & 1.0612018\tabularnewline
Movie effect model & 0.9439087\tabularnewline
\bottomrule
\end{longtable}

So we have predicted movie rating based on the fact that movies are
rated differently by adding the computed \(b_{i}\) to \(\mu\). If an
individual movie is on average rated worse than the average rating of
all movies \(\mu\), we predict that it will rate lower than \(\mu\) by
\(b_{i}\), the difference of the individual movie average from the total
average.

We can see an improvement, but this can be further improved by consider
the individual user rating effect.

\pagebreak

\hypertarget{movie-and-user-effect-model}{%
\subsection{Movie and User effect
model}\label{movie-and-user-effect-model}}

We compute the average rating for user \(\mu\), for those that have
rated atleast 100 movies. In fact users affect the ratings positively or
negatively. So, We incorporate the user effect term to our prediction.

\begin{Shaded}
\begin{Highlighting}[]
\CommentTok{# Compute the bias terms of user ratings}
\NormalTok{user_avgs<-}\StringTok{ }\NormalTok{edx }\OperatorTok\StringTok{ }\KeywordTok{left_join}\NormalTok{(movie_avgs, }\DataTypeTok{by=}\StringTok{'movieId'}\NormalTok{) }\OperatorTok
\StringTok{                    }\KeywordTok{group_by}\NormalTok{(userId) }\OperatorTok
\StringTok{                    }\KeywordTok{filter}\NormalTok{(}\KeywordTok{n}\NormalTok{() }\OperatorTok{>=}\StringTok{ }\DecValTok{100}\NormalTok{) }\OperatorTok
\StringTok{                    }\KeywordTok{summarize}\NormalTok{(}\DataTypeTok{b_u =} \KeywordTok{mean}\NormalTok{(rating }\OperatorTok{-}\StringTok{ }\NormalTok{mu }\OperatorTok{-}\StringTok{ }\NormalTok{b_i))}
\CommentTok{# plot distribution}
\NormalTok{user_avgs}\OperatorTok\StringTok{ }\KeywordTok{qplot}\NormalTok{(b_u, }\DataTypeTok{geom =}\StringTok{"histogram"}\NormalTok{, }\DataTypeTok{bins =} \DecValTok{30}\NormalTok{, }\DataTypeTok{data =}\NormalTok{ ., }\DataTypeTok{color =} \KeywordTok{I}\NormalTok{(}\StringTok{"#51A8C9"}\NormalTok{),}
                   \DataTypeTok{main =} \StringTok{"Number of users for respective computed bias"}\NormalTok{)  }\OperatorTok{+}\StringTok{ }
\StringTok{             }\KeywordTok{scale_colour_wsj}\NormalTok{(}\StringTok{"colors6"}\NormalTok{, }\StringTok{""}\NormalTok{) }\OperatorTok{+}
\StringTok{             }\KeywordTok{theme_wsj}\NormalTok{(}\DataTypeTok{base_size =} \DecValTok{5}\NormalTok{, }\DataTypeTok{color =} \StringTok{"blue"}\NormalTok{, }
                        \DataTypeTok{base_family =} \StringTok{"sans"}\NormalTok{, }\DataTypeTok{title_family =} \StringTok{"sans"}\NormalTok{)}
\end{Highlighting}
\end{Shaded}

\includegraphics{Movielens_Predictions_files/figure-latex/unnamed-chunk-8-1.pdf}

We can see an improvement, but this model does not consider the
individual user rating effect. So we can improve our model by
incorporating user rating effect:

\[Y_{u, i} = \mu + b_{i} + b_{u} + \epsilon_{u, i}\] We can see an
improvement, but this model does not consider the individual user rating
effect. So we can improve our model by incorporating user rating effect:
where \(b_{u}\) is a user-specific effect. If a choosy user (negative
\(b_{u}\) rates a great movie (positive \(b_{i}\)), the effects counter
each other, and we may be able to correctly predict that this user gave
this great movie a three rather than a five.

We compute an approximation by computing \(\mu\) and \(b_{i}\), and
estimating \(b_{u}\), as the average of \[Y_{u, i} - \mu - b_{i}\]

\begin{Shaded}
\begin{Highlighting}[]
\CommentTok{# Compute user averages}
\NormalTok{user_avgs <-}\StringTok{ }\NormalTok{edx }\OperatorTok
\StringTok{  }\KeywordTok{left_join}\NormalTok{(movie_avgs, }\DataTypeTok{by=}\StringTok{'movieId'}\NormalTok{) }\OperatorTok
\StringTok{  }\KeywordTok{group_by}\NormalTok{(userId) }\OperatorTok
\StringTok{  }\KeywordTok{summarize}\NormalTok{(}\DataTypeTok{b_u =} \KeywordTok{mean}\NormalTok{(rating }\OperatorTok{-}\StringTok{ }\NormalTok{mu }\OperatorTok{-}\StringTok{ }\NormalTok{b_i))}
\end{Highlighting}
\end{Shaded}

We can now construct predictors and see that the RMSE improves:

\begin{Shaded}
\begin{Highlighting}[]
\CommentTok{# predictions for this model}
\NormalTok{predictions <-}\StringTok{ }\NormalTok{validation}\OperatorTok
\StringTok{               }\KeywordTok{left_join}\NormalTok{(movie_avgs, }\DataTypeTok{by=}\StringTok{'movieId'}\NormalTok{) }\OperatorTok
\StringTok{               }\KeywordTok{left_join}\NormalTok{(user_avgs, }\DataTypeTok{by=}\StringTok{'userId'}\NormalTok{) }\OperatorTok
\StringTok{               }\KeywordTok{mutate}\NormalTok{(}\DataTypeTok{pred =}\NormalTok{ mu }\OperatorTok{+}\StringTok{ }\NormalTok{b_i }\OperatorTok{+}\StringTok{ }\NormalTok{b_u) }\OperatorTok
\StringTok{               }\KeywordTok{pull}\NormalTok{(pred)}

\NormalTok{user_effect_rmse <-}\StringTok{ }\KeywordTok{RMSE}\NormalTok{(predictions, validation}\OperatorTok{$}\NormalTok{rating)}
\NormalTok{rmse_results <-}\StringTok{ }\KeywordTok{bind_rows}\NormalTok{(rmse_results,}
                          \KeywordTok{tibble}\NormalTok{(}\DataTypeTok{Model=}\StringTok{"Movie & User effect model"}\NormalTok{,  }
                                     \DataTypeTok{RMSE =}\NormalTok{ user_effect_rmse))}
\NormalTok{rmse_results }\OperatorTok\StringTok{ }\NormalTok{knitr}\OperatorTok{::}\KeywordTok{kable}\NormalTok{()}
\end{Highlighting}
\end{Shaded}

\begin{longtable}[]{@{}lr@{}}
\toprule
Model & RMSE\tabularnewline
\midrule
\endhead
Naïve Average movie rating model & 1.0612018\tabularnewline
Movie effect model & 0.9439087\tabularnewline
Movie \& User effect model & 0.8653488\tabularnewline
\bottomrule
\end{longtable}

\pagebreak

Our rating predictions further reduced the RMSE. However, the purported
best and worst movie were rated by a few users, in most cases, just one
user. These movies were mostly obscure ones. This fact is because, with
a few users, we have more uncertainty. Therefore larger estimates of
\(b_{i}\), negative or positive, are more likely.

Until now, we computed standard error and constructed confidence
intervals to account for different levels of uncertainty. However, when
making predictions, we need one number, one prediction, not an interval.
For this, we introduce the concept of regularization, that permits to
penalize high estimates that come from small sample sizes. The general
idea is to add a penalty for large values of \(b_{i}\) to the cost
function(RMSE) that we want to minimize. So having many large \(b_{i}\),
makes it harder to minimize. Regularization is a method used to reduce
the effect of overfitting.

\hypertarget{iv.-regularized-movie-and-user-effect-model}{%
\subsection{IV. Regularized Movie and User effect
model}\label{iv.-regularized-movie-and-user-effect-model}}

So estimates of \(b_{i}\) and \(b_{u}\) are caused by movies with very
few ratings and in some users that only rated a tiny number of movies.
Hence this can strongly influence the prediction---the use of
regularization permits to penalize these aspects. We should find the
value of lambda (regularization term) that will minimize the RMSE. This
shrinks the \(b_{i}\) and \(b_{u}\) in case of a small number of
ratings.

\begin{Shaded}
\begin{Highlighting}[]
\CommentTok{# Trying different values for the regularization term}
\NormalTok{lambdas <-}\StringTok{ }\KeywordTok{seq}\NormalTok{(}\DecValTok{2}\NormalTok{, }\DecValTok{10}\NormalTok{, }\FloatTok{0.25}\NormalTok{)}
\NormalTok{rmses <-}\StringTok{ }\KeywordTok{sapply}\NormalTok{(lambdas, }\ControlFlowTok{function}\NormalTok{(lambda)\{}
  
\NormalTok{  movie_avg <-}\StringTok{ }\KeywordTok{mean}\NormalTok{(edx}\OperatorTok{$}\NormalTok{rating)}
  
\NormalTok{  b_i <-}\StringTok{ }\NormalTok{edx }\OperatorTok\StringTok{ }
\StringTok{    }\KeywordTok{group_by}\NormalTok{(movieId) }\OperatorTok
\StringTok{    }\KeywordTok{summarize}\NormalTok{(}\DataTypeTok{b_i =} \KeywordTok{sum}\NormalTok{(rating }\OperatorTok{-}\StringTok{ }\NormalTok{mu)}\OperatorTok{/}\NormalTok{(}\KeywordTok{n}\NormalTok{()}\OperatorTok{+}\NormalTok{lambda))}
  
\NormalTok{  b_u <-}\StringTok{ }\NormalTok{edx }\OperatorTok\StringTok{ }
\StringTok{    }\KeywordTok{left_join}\NormalTok{(b_i, }\DataTypeTok{by=}\StringTok{"movieId"}\NormalTok{) }\OperatorTok
\StringTok{    }\KeywordTok{group_by}\NormalTok{(userId) }\OperatorTok
\StringTok{    }\KeywordTok{summarize}\NormalTok{(}\DataTypeTok{b_u =} \KeywordTok{sum}\NormalTok{(rating }\OperatorTok{-}\StringTok{ }\NormalTok{b_i }\OperatorTok{-}\StringTok{ }\NormalTok{mu)}\OperatorTok{/}\NormalTok{(}\KeywordTok{n}\NormalTok{()}\OperatorTok{+}\NormalTok{lambda))}
  
\NormalTok{  predicted_ratings <-}\StringTok{ }
\StringTok{    }\NormalTok{validation }\OperatorTok\StringTok{ }
\StringTok{    }\KeywordTok{left_join}\NormalTok{(b_i, }\DataTypeTok{by =} \StringTok{"movieId"}\NormalTok{) }\OperatorTok
\StringTok{    }\KeywordTok{left_join}\NormalTok{(b_u, }\DataTypeTok{by =} \StringTok{"userId"}\NormalTok{) }\OperatorTok
\StringTok{    }\KeywordTok{mutate}\NormalTok{(}\DataTypeTok{pred =}\NormalTok{ mu }\OperatorTok{+}\StringTok{ }\NormalTok{b_i }\OperatorTok{+}\StringTok{ }\NormalTok{b_u) }\OperatorTok
\StringTok{    }\KeywordTok{pull}\NormalTok{(pred)}
  
  \KeywordTok{return}\NormalTok{(}\KeywordTok{RMSE}\NormalTok{(predicted_ratings, validation}\OperatorTok{$}\NormalTok{rating))}
\NormalTok{\})}
\end{Highlighting}
\end{Shaded}

We plot RMSE vs lambdas to select the optimal lambda:

\begin{Shaded}
\begin{Highlighting}[]
\CommentTok{# Plot rmses vs lambdas to select the optimal omega                                                             }
\CommentTok{# Plot rmses vs lambdas to select the optimal omega                                                             }
\KeywordTok{qplot}\NormalTok{(lambdas, rmses,  }\DataTypeTok{color =} \KeywordTok{I}\NormalTok{(}\StringTok{"#51A8FF"}\NormalTok{), }
      \DataTypeTok{main =} \StringTok{"RMSEs vs. Lambdas"}\NormalTok{) }\OperatorTok{+}\StringTok{ }
\StringTok{      }\KeywordTok{theme_fivethirtyeight}\NormalTok{()}
\end{Highlighting}
\end{Shaded}

\includegraphics{Movielens_Predictions_files/figure-latex/plot_omegas-1.pdf}

For the final model, the optimal lambda is:

\begin{Shaded}
\begin{Highlighting}[]
\CommentTok{# get the lamba for minimum value of rmse}
\NormalTok{lambda <-}\StringTok{ }\NormalTok{lambdas[}\KeywordTok{which.min}\NormalTok{(rmses)]}
\NormalTok{lambda}
\end{Highlighting}
\end{Shaded}

\begin{verbatim}
[1] 5.25
\end{verbatim}

For the final model, the optimal lambda(\(\lambda\)) is \texttt{lambda}.

The final results are:

\begin{Shaded}
\begin{Highlighting}[]
\CommentTok{# Test and save results    }
\NormalTok{rmse_results <-}\StringTok{ }\KeywordTok{bind_rows}\NormalTok{(rmse_results,}
                          \KeywordTok{tibble}\NormalTok{(}\DataTypeTok{Model=}\StringTok{"Regularisation & Movie & User effect model"}\NormalTok{,  }
                                     \DataTypeTok{RMSE =} \KeywordTok{min}\NormalTok{(rmses)))}
\NormalTok{rmse_results }\OperatorTok\StringTok{ }\NormalTok{knitr}\OperatorTok{::}\KeywordTok{kable}\NormalTok{()}
\end{Highlighting}
\end{Shaded}

\begin{longtable}[]{@{}lr@{}}
\toprule
Model & RMSE\tabularnewline
\midrule
\endhead
Naïve Average movie rating model & 1.0612018\tabularnewline
Movie effect model & 0.9439087\tabularnewline
Movie \& User effect model & 0.8653488\tabularnewline
Regularisation \& Movie \& User effect model & 0.8648170\tabularnewline
\bottomrule
\end{longtable}

\pagebreak

\hypertarget{results}{%
\section{Results}\label{results}}

The RMSE values of all the represented models: \#\# The Result

\begin{longtable}[]{@{}lr@{}}
\toprule
Model & RMSE\tabularnewline
\midrule
\endhead
Naïve Average movie rating model & 1.0612018\tabularnewline
Movie effect model & 0.9439087\tabularnewline
Movie \& User effect model & 0.8653488\tabularnewline
Regularisation \& Movie \& User effect model & 0.8648170\tabularnewline
\bottomrule
\end{longtable}

We find the lowest value of RMSE to be at 0.8648170

\hypertarget{conclusion}{%
\section{Conclusion}\label{conclusion}}

The final model for this project is:

\[Y_{u, i} = \mu + b_{i} + b_{u} + \epsilon_{u, i}\]

This model works well if the average user doesn't rate a particularly
good/popular movie with a large positive \(b_{i}\), and vice versa.

We have built multiple machine learning algorithms to predict movie
ratings using the MovieLens dataset with increasing complexity to lower
the RMSE. The regularized model, including the effect of the users, is
characterized by the lower RMSE value and is hence the optimal model to
use for the present project. With this model, we have achieved our goal
of creating an algorithm with RMSE(0.8648170) lower than the allowed
RMSE(0.8649). Further, RMSE could be improved by adding other effects
(genre, year, age,..). Applying different machine learning models could
also improve the results further. The models build on the assumptions
that also the movie popularity, the user opinion and genre popularity
are constant over time. Further work could be done to investigate if
this is true, and if not, the changes over time could be built into the
models.

\pagebreak

\end{document}
